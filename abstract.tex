%%%%%%%%%%%%%%%%%%%%%%%%%%%%%%%%%%%%%%%%%%%%%%%%%%%%%%%%%%%%%%%%%%%%%%%%%%%%%%%%%%%%%%%%%%%%%%%%%%%%%%
%
%   Filename    : abstract.tex 
%
%   Description : This file will contain your abstract.
%                 
%%%%%%%%%%%%%%%%%%%%%%%%%%%%%%%%%%%%%%%%%%%%%%%%%%%%%%%%%%%%%%%%%%%%%%%%%%%%%%%%%%%%%%%%%%%%%%%%%%%%%%

\begin{abstract}
% From 150 to 200 words of short, direct and complete sentences, the abstract 
% should be informative enough to serve as a substitute for reading the thesis document 
% itself.  It states the rationale and the objectives of the research.  

% In the final thesis document (i.e., the document you'll submit for your final thesis defense), the 
% abstract should also contain a description of your research results, findings, 
% and contribution(s).

This research explores the interaction among children learning through a playful mobile musical interface.  In using the innate playful nature of children, a sandbox environment will be implemented to supplement the experiences of learning music. We will design gestures that allows children to interact with a firefly model that represents different musical rudiments, such as rhythm, beat-rest patterns, notes, measures, and sections. This will be accomplished through the use of a human-centered design process of understanding children, and music teachers. The findings will guide the design and development of an iterative prototype that will repeatedly be tested by children. Continuous feedback through experiments and usability tests will allow us to discover what human factors are exhibited by children when doing music composition tasks.

%
%  Do not put citations or quotes in the abstract.
%

% Keywords can be found at \url{http://www.acm.org/about/class/class/2012?pageIndex=0}.  Click the 
% link ``HTML'' in the paragraph that starts with ''The \textbf{full CCS classification tree}...''.

\begin{flushleft}
\begin{tabular}{lp{4.25in}}
\hspace{-0.5em}\textbf{Keywords:}\hspace{0.25em} & Human Computer Interaction, Music Representation, Sandbox Environment, Gestural Input, Usability testing, User Interface Design
\end{tabular}
\end{flushleft}
\end{abstract}
